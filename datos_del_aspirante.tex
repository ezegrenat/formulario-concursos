%% Datos personales
%% ----------------

\newcommand{\nombresDelAspirante}{Ezequiel}
\newcommand{\apellidoDelAspirante}{Grenat}
\newcommand{\numeroDocumentoDeIdentidadDelAspirante}{42567191}
\newcommand{\numeroDiaNacimientoDelAspirante}{22}
\newcommand{\numeroMesNacimientoDelAspirante}{03}
\newcommand{\anioNacimientoDelAspirante}{2000}
\newcommand{\telefonoDelAspirante}{11-3059-4569}
\newcommand{\emailDelAspirante}{ezequielggrenat@gmail.com}

% Opcional: Email alternativo. Si no tenés ninguno, dejalo en blanco.
\newcommand{\emailAlternativoDelAspirante}{}

% Completar únicamente si lo tuviera. Si no, dejarlo en blanco.
  % El número de legajo te lo asignan cuando empezás a trabajar para la facultad.
  % Si nunca trabajaste para la facultad seguramente no tengas así que no tenés que completarlo.
\newcommand{\numeroDeLegajoDelAspirante}{}

%% Concurso
%% --------

% Descomentar la línea que corresponda
% \newcommand{\JTP}{} % Jefe/a de trabajos prácticos
% \newcommand{\AYp}{} % Ayudante de 1°
\newcommand{\AYs}{} % Ayudante de 2°

% Descomentar la línea que corresponda (para Ay2 es siempre Parcial)
\newcommand{\Parcial}{} % Parcial
% \newcommand{\Semi}{} % SemiExclusiva
% \newcommand{\Exclu}{} % Exclusiva

% Área a concursar "Algoritmos", "Sistemas" o "Teoría"
\newcommand{\area}{Algoritmos}

% Completar si te tomaste alguna licencia
% Si estás concursando por primera vez, dejalo como está
\newcommand{\licencias}{No se tuvo ninguna licencia.}

% Definir si querés que tengan en cuenta tu desempeño docente durante la pandemia (si no querés comentá la definición del comando)
% Si estás concursando por primera vez, dejalo como está
\newcommand{\considerameLaPandemia}{}

% Si tenés una imagen con tu firma, descomentá la siguiente línea para que se agregue en cada carilla:
% \newcommand{\firma}{firma.jpg}
% Si la imagen es muy chica o muy grande, modificá a continuación los valores de escala:
\newcommand{\escalaFirmaPrincipal}{0.05}
\newcommand{\escalaFirmaCadaCarilla}{0.16}
